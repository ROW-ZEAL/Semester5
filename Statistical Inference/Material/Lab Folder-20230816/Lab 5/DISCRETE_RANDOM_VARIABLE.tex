\documentclass{article}\usepackage[]{graphicx}\usepackage[]{xcolor}
% maxwidth is the original width if it is less than linewidth
% otherwise use linewidth (to make sure the graphics do not exceed the margin)
\makeatletter
\def\maxwidth{ %
  \ifdim\Gin@nat@width>\linewidth
    \linewidth
  \else
    \Gin@nat@width
  \fi
}
\makeatother

\definecolor{fgcolor}{rgb}{0.345, 0.345, 0.345}
\newcommand{\hlnum}[1]{\textcolor[rgb]{0.686,0.059,0.569}{#1}}%
\newcommand{\hlstr}[1]{\textcolor[rgb]{0.192,0.494,0.8}{#1}}%
\newcommand{\hlcom}[1]{\textcolor[rgb]{0.678,0.584,0.686}{\textit{#1}}}%
\newcommand{\hlopt}[1]{\textcolor[rgb]{0,0,0}{#1}}%
\newcommand{\hlstd}[1]{\textcolor[rgb]{0.345,0.345,0.345}{#1}}%
\newcommand{\hlkwa}[1]{\textcolor[rgb]{0.161,0.373,0.58}{\textbf{#1}}}%
\newcommand{\hlkwb}[1]{\textcolor[rgb]{0.69,0.353,0.396}{#1}}%
\newcommand{\hlkwc}[1]{\textcolor[rgb]{0.333,0.667,0.333}{#1}}%
\newcommand{\hlkwd}[1]{\textcolor[rgb]{0.737,0.353,0.396}{\textbf{#1}}}%
\let\hlipl\hlkwb

\usepackage{framed}
\makeatletter
\newenvironment{kframe}{%
 \def\at@end@of@kframe{}%
 \ifinner\ifhmode%
  \def\at@end@of@kframe{\end{minipage}}%
  \begin{minipage}{\columnwidth}%
 \fi\fi%
 \def\FrameCommand##1{\hskip\@totalleftmargin \hskip-\fboxsep
 \colorbox{shadecolor}{##1}\hskip-\fboxsep
     % There is no \\@totalrightmargin, so:
     \hskip-\linewidth \hskip-\@totalleftmargin \hskip\columnwidth}%
 \MakeFramed {\advance\hsize-\width
   \@totalleftmargin\z@ \linewidth\hsize
   \@setminipage}}%
 {\par\unskip\endMakeFramed%
 \at@end@of@kframe}
\makeatother

\definecolor{shadecolor}{rgb}{.97, .97, .97}
\definecolor{messagecolor}{rgb}{0, 0, 0}
\definecolor{warningcolor}{rgb}{1, 0, 1}
\definecolor{errorcolor}{rgb}{1, 0, 0}
\newenvironment{knitrout}{}{} % an empty environment to be redefined in TeX

\usepackage{alltt}

\title{Discrete Random Variable}
\author{Thulasy}

\date{}
\IfFileExists{upquote.sty}{\usepackage{upquote}}{}
\begin{document}
\maketitle{}

\begin{enumerate}

\item What name is given to a table that lists all the values that a discrete random variable x can assume and their corresponding probabilities?\\
\textbf{Answer}\\
Probability Distribution Table

\item Classify each of the following random variables as discrete or continuous.\\
\begin{enumerate}
\item The time left on a parking meter\\
\textbf{Answer}\\
Continuous Variable\\
\item The total pounds of fish caught on a fishing trip\\
\textbf{Answer}\\
Continuous Variable\\
\item The number of gumballs in a vending machine\\
\textbf{Answer}\\
Discrete Variable\\
\item The time spent by a physician examining a patient\\
\textbf{Answer}\\
Continuous Variable\\
\end{enumerate}

\item A household can watch news on any of the three networks ??? ABC, CBS, or NBC. On a certain day, five households randomly and independently decide which channel to watch. Let x be the number of households among these five that decide to watch news on ABC. Is x a discrete or a continuous random variables? Explain.\\
\textbf{Answer}\\
x = the number of households among these five that decide to watch news on ABC\\
x=0,1,2,3,4,5\\
x is a discrete random variable because the number of family cannot be a real number\\

\item The following table gives the probability distribution of a discrete random variable x.

\begin{tabular}{cc}
x & p(x)\\
0 & 0.11\\
1 & 0.19\\
2 & 0.28\\
3 & 0.15\\
4 & 0.12\\
5 & 0.09\\
6 & 0.06\\
\end{tabular}
\\
Find the probabilities?\\
\begin{enumerate}
\item P(x=3)	\\
\textbf{Answer}\\
P(x=3)=0.15
\item P(x $\leq$ 2) \\
\textbf{Answer}\\
P(x$\leq$2)=P(x=0)+P(x=1)+P(x=2)=0.11+0.19+0.28\\
\item P(x $\geq$ 3)\\
P(x $\geq$ 3) = P(x=3)+P(x+4)+P(x=5)+P(x=6)=0.15+0.12+0.09+0.06\
\item $P(1 \leq x \leq 4)$ = P(x=1)+P(x=2)+P(x=3)+P(x=4)\\
e.	Probability that x assumes a value less than 4
f.	Probability that x assumes a value greater than 2
g.	Probability that x assumes  a value in the interval 2 to 5
h.	Probability that x assumes  a value between 2 to 5


\end{enumerate}

\end{enumerate}

\end{document}
